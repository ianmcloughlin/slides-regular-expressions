\documentclass[dvipsnames,hidelinks,t]{beamer}

  % Enables the use of colour. 
  \usepackage{xcolor}
  % Syntax high-lighting for code. Requires Python's pygments.
  \usepackage{minted}
  % Enables the use of umlauts and other accents.
  \usepackage[utf8]{inputenc}
  % Diagrams.
  \usepackage{tikz}
  % Settings for captions, such as sideways captions.
  \usepackage{caption}
  % Symbols for units, like degrees and ohms.
  \usepackage{gensymb}
  % Latin modern fonts - better looking than the defaults.
  \usepackage{lmodern}
  % Allows for columns spanning multiple rows in tables.
  \usepackage{multirow}
  % Better looking tables, including nicer borders.
  \usepackage{booktabs}
  % More math symbols.
  \usepackage{amssymb}
  % More math fonts, like mathbb.
  \usepackage{amsfonts}
  % More math layouts, equation arrays, etc.
  \usepackage{amsmath}
  % More theorem environments.
  \usepackage{amsthm}
  % More column formats for tables.
  \usepackage{array}
  % Adjust the sizes of box environments.
  \usepackage{adjustbox}
  % Better looking single quotes in verbatim and minted environments.
  \usepackage{upquote}
  % Better blank space decisions.
  \usepackage{xspace}
  % Better looking tikz trees.
  \usepackage{forest}
  % URLs.
  \usepackage{hyperref}
  % For plotting.
  \usepackage{pgfplots}
  
  % Various tikz libraries.
  % For drawing mind maps.
  \usetikzlibrary{mindmap}
  % For adding shadows.
  \usetikzlibrary{shadows}
  % Extra arrows tips.
  \usetikzlibrary{arrows.meta}
  % Old arrows.
  \usetikzlibrary{arrows}
  % Automata.
  \usetikzlibrary{automata}
  % For more positioning options.
  \usetikzlibrary{positioning}
  % Creating chains of nodes on a line.
  \usetikzlibrary{chains}
  % Fitting node to contain set of coordinates.
  \usetikzlibrary{fit}
  % Extra shapes for drawing.
  \usetikzlibrary{shapes}
  % For markings on paths.
  \usetikzlibrary{decorations.markings}
  % For advanced calculations.
  \usetikzlibrary{calc}
  
  % GMIT colours.
  \definecolor{gmitblue}{RGB}{20,134,225}
  \definecolor{gmitred}{RGB}{220,20,60}
  \definecolor{gmitgrey}{RGB}{67,67,67}
  
  % Change some style options.
  \usetheme{metropolis}
  % Tell minted to use the following colour scheme. 
  \usemintedstyle{manni}
  % Remove some of the vertical space after the title.
  \addtobeamertemplate{frametitle}{}{\vspace{-3mm}}
  % Change the default theme colours.
  \setbeamercolor{normal text}{fg=darkgray, bg=white}
  \setbeamercolor{alerted text}{fg=gmitred, bg=white}
  \setbeamercolor{example text}{fg=gmitblue, bg=white}
  \setbeamercolor{frametitle}{fg=gmitblue, bg=white}
  \setbeamercolor*{item}{fg=gmitblue}
  % Use a better math mode font.
  \usefonttheme[onlymath]{serif}
  % Don't display section pages.
  \metroset{sectionpage=none}
  % Change the default itemize bullets.
  \setbeamertemplate{itemize item}{\color{gray}--}
  % Change the position of left aligned math.
  %\setlength{\mathindent}{7mm}

  % An environment for displaying math in red, without lots of vertical space.
  \newcommand{\redmath}[1]{\vspace{-3mm} {\begin{center} \color{gmitred} $ #1 $ \end{center}} \vspace{-2mm}}

  % For displaying a blank character.
  \newcommand{\bl}{\underline{\hspace{2mm}}}

  % \citeurl can be used to a clickable short url to a slide as a reference.
  \renewcommand\footnoterule{}
  \newcommand{\citeurl}[1]{\let\thefootnote\relax\footnotetext{\tiny \textcolor{gmitgrey}{\href{http://#1}{#1}}}}
  \newcommand{\citeeg}[1]{\let\thefootnote\relax\footnotetext{\tiny \textcolor{gmitgrey}{#1}}}
  
  % Prevent minted from showing errors.
  \makeatletter
  \expandafter\def\csname PYGdefault@tok@err\endcsname{\def\PYGdefault@bc##1{{\strut ##1}}}
  \makeatother
  
  \begin{document}
    \title{Regular expressions}
    \subtitle{}
    \author{ian.mcloughlin@gmit.ie}
    \date{}
  
    \begin{frame}
      \titlepage
    \end{frame}
  
    \begin{frame}{Regular expressions}
  
  \redmath{(0.0|1.1).(0.1)^*.(0.0|1.1)}

  \begin{itemize}
    \item Regular expressions are strings that represent patterns of text.
    \item The strings can contain special characters.
    \item Brackets can be used to group characters together.
    \item Regular expressions are used to search other strings for patterns.
  \end{itemize}
  \begin{alertblock}{Special characters}
    \begin{description}[abbb]
      \item[$.$] means \emph{concatenate}. So, $a.b$ means an $a$ followed by a $b$.
      \item[$|$] means \emph{or}. So, $a|b$ means an $a$ or a $b$.
      \item[$*$] means \emph{zero or more times}. So, $a^*$ means zero or more $a$'s.
    \end{description}
  \end{alertblock}
\end{frame}

\begin{frame}{Examples of regular expressions}
  \begin{description}[aaaaaaaa]
    \setlength\itemsep{5mm}
    \item[$a.b.c$] -- a single $a$ followed by a single $b$ followed by a single $c$. 
    \item[$a.b.c^*$] -- an $a$ followed by a $b$ followed by zero or more $c$'s.
    \item[$a|b.c$] -- an $a$, or a $b$ followed by a $c$.
    \item[$(a|b).c$] -- an $a$ or a $b$, followed by a $c$.
    \item[$0.0.(0|1)^*$] -- all strings of $0$'s and $1$' that begin with two zeros.
    \item[$1^*$] -- any number of $1$'s (including empty string).
  \end{description}
\end{frame}


\begin{frame}{Precedence}
  \begin{enumerate}
    \setlength\itemsep{5mm}
    \item Always apply $^*$ first.
    \item Apply $.$ after $^*$ but before $|$.
    \item Apply $|$ last.
    \item Treat bracketed groups as individual characters.
  \end{enumerate}
\end{frame}


\begin{frame}{Languages of regular expressions}
  This is a list of regular expressions over an alphabet $A$ and the languages they define.
  \begin{description}[aa]
    \setlength\itemsep{4mm}
    \item[$\emptyset$] -- the empty language ($\{\}$).
    \item[$\epsilon$] -- the language containing the empty string ($\{\epsilon\})$.
    \item[$a$] -- the language containing any symbol $a$ from $A$ ($\{a\}$).
    \item[$r.s$] -- the concatenation of the languages defined by regular expressions $r$ and $s$.
    \item[$r|s$] -- the union of the languages of regular expressions $r$ and $s$.
    \item[$r^*$] -- the star of the language defined by regular expression $r$.
  \end{description}
\end{frame}


\begin{frame}{Infix and postfix}
  It is sometimes convenient to re-write expressions in postfix.
  This applies to lots of different expressions, not just regular expressions.
  \begin{exampleblock}{Example}
    Convert an infix mathematical expression (left) to postfix (right):
    $$(3+4) \times 5 \qquad \rightarrow \qquad 3 \ 4 \  + 5 \  \times$$
  \end{exampleblock}
  \begin{exampleblock}{Example}
    Converting an infix regular expression (left) to postfix (right):
    $$a.(b.b)^*.a \qquad  \rightarrow  \qquad  abb.^*.a.$$
    Note we often omit the $.$ in infix notation: ``$a(bb)^*a$'' but can't in postfix.
    However, the brackets aren't needed in postfix.
  \end{exampleblock}
\end{frame} 
  \end{document}