 \begin{frame}{Regular expressions}
  \begin{columns}[T]
    \begin{column}{0.5\textwidth}
      \begin{itemize}
        \setlength\itemsep{5mm}
        \item Regular expressions are strings that represent patterns of text.
        \item The strings can contain special characters.
        \item Brackets can be used to group characters together.
        \item Regular expressions are used to search other strings for patterns.
      \end{itemize}
    \end{column}
\vrule{}
    \begin{column}{0.5\textwidth}
      \begin{exampleblock}{Examples}
        \redmathl{a.b.c^*} \\
        An $a$ followed by a $b$ followed by zero or more $c$'s. \\
        \redmathl{a|b.c} \\
        An $a$, or a $b$ followed by a $c$. \\
        \redmathl{(a|b).c} \\
        An $a$ or a $b$, followed by a $c$. \\
        \redmathl{0.0.(0|1)^*}\\
        All strings of $0$'s and $1$' that begin with two zeros.
      \end{exampleblock}
    \end{column}
  \end{columns}
\end{frame}


\begin{frame}{Special characters}
  \begin{columns}
    \begin{column}{0.5\textwidth}
      \begin{description}[abbb]
        \item[$.$] means \emph{concatenate}. So, $a.b$ means an $a$ followed by a $b$.
        \item[$|$] means \emph{or}. So, $a|b$ means an $a$ or a $b$.
        \item[$^*$] means \emph{zero of more times}. So, $a^*$ means zero or more $a$'s.
      \end{description}
    \end{column}
    \begin{column}{0.5\textwidth}
      \begin{alertblock}{Precedence}
        \begin{enumerate}
          \setlength\itemsep{5mm}
          \item Always apply $^*$ first.
          \item Apply $.$ after $^*$ but before $|$.
          \item Apply $|$ last.
          \item Treat bracketed groups as individual characters.
        \end{enumerate}
      \end{alertblock}
    \end{column}
  \end{columns}
\end{frame}

\begin{frame}{Infix and postfix}
  It is sometimes convenient to re-write expressions in postfix.
  This applies to lots of different expressions, not just regular expressions.
  \vspace{2mm}
  \begin{exampleblock}{Example}
    The infix mathematical expression ``$(3+4) \times 5$'' is ``$3 \ 4 \ + 5 \ \times$'' in postfix.
  \end{exampleblock}
  \vspace{2mm}
  \begin{exampleblock}{Example}
    The infix regular expression ``$a.(b.b)^*.a$'' is ``$abb.^*.a.$'' in postfix.
    Note we often omit the $.$ in infix notation: ``$a(bb)^*a$'' but can't in postfix.
    However, the brackets aren't needed in postfix.
  \end{exampleblock}
\end{frame}

\begin{frame}{Executing regular expressions against strings}
  \redmath{r = ``(0^*1^*|1^*0^*)''}
  \redmath{s = ``00000000011111111111111111''}

  \begin{itemize}
    \item Regular expressions are \emph{executed} against strings.
    \item This means an algorithm determines if the string ($s$) matches the pattern as defined by the regular expression ($r$).
    \item We can ask two related questions: does the whole string match, or does a substring of it match?
    \item We can write algorithms to answer either question.
  \end{itemize}
\end{frame}